Para a gestão das torres e das keys existem duas classes, \textit{Towers.cpp} e \textit{Keys.cpp} respectivamente.

\-
\begin{center}
\begin{tabular} {l | p{10cm}}
\begin{lstlisting}
class Keys {
public:
    int num_keys, catched_count;
    float catch_dist;
    Key **keys;

    Keys();

    void render();
    void update();
    int get_closest_distance();
};
\end{lstlisting} 
& 
Header da classe Keys.cpp .
\end{tabular}
\end{center}
\-
Como se pode ver a classe keys é constituida por um array de instâncias da classe \emph{Key}, um numero de chaves (\emph{num\_keys}) que se encontra no documento config.ini e o numero de chaves apanhadas pela jogador (\emph{catched\_count}).
Existe ainda um método de update, que actualiza o modelo das chaves, determina se uma chave foi ou não apanhada, um método de render, que desenha as torres no mundo e por ultimo um método que calcula a distancia da chave que se encontra a menor distancia do jogador, inferior a 500 metros.

Para as torres o conceito é o mesmo, com a excepção do \emph{catched\_count} e do método \emph{get\_closest\_distance}.
No entanto as torres têm a caracteristaca de rodar na direcção do jogador quando este se encontra no seu alcance.

Na inicialização do jogo, são calculadas posições aleatórias para as chaves e torres. Estas posições têm em conta que uma torre ou chave nunca deverá ficar a menos de uma determinada distância (definida no ficheiro ini) de outra chave ou torre, evitando assim por exemplo, que uma chave fique escondida dentro do modelo de uma torre.

O algoritmo de cálculo da {\bf distancia entre dois modelos} é simples. Tendo as coordenadas do dos modelos temos:

\begin{center}
\begin{math}
Player => (a1,b1)
\end{math}
,
\begin{math}
Tower => (a2,b2)
\end{math}

\begin{equation}
Distancia = \sqrt{((a1-a2)^2+(b1-b2)^2)}
\end{equation}
\end{center}


\subsection{Angulo de orientação das Torres}

O angulo de orientação das torres para o player pode ser determinado fazendo uso do produto escalar entre dois vectores, a direção do player para a torre e o vector unitário (1,0,0). Sabemos que

\-
\begin{center}
\begin{math}
Player => (a_1,a_2)
\end{math}
,
\begin{math}
Tower => (b_1,b_2)
\end{math}

\begin{equation}
produto escalar = a_1*b_1 + a_2*b_2
\end{equation}
\begin{equation}
produto escalar = direccao * unitario * cos x
\end{equation}
\end{center}

Como tal, orientado em função do angulo \emph{x} temos:

\begin{equation}
x = acos( (a_1*b_1 + a_2*b_2 ) / direccao * unitario * cos x )
\end{equation}


Este angulo é a orientação que uma torre deve assumir para se orientar para um player, está claro, quando a distância é inferior ao alcance da torre.




