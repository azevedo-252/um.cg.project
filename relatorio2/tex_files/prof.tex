\chapter{Profiling}
Ainda que um extra, ou talvez mesmo não um extra uma vez que é uma componente de desenvolvimento não de jogo, cramos também um modulo para profiling. Para efectuar os calculos existem três arrays para guardar a informação:

\begin{description}
\item[\textit{start}] Registo o tempo no inicio da medição.
\item[\textit{end}] Regista o tempo no final da medição.
\item[\textit{name}] Regista a descrição da medição.
\end{description}

A utilização é intuitiva. É invocado o método start\_time com um identificador e uma descrição, ou nome no inicio da medição e end\_time com o mesmo identificador no final da medição. Feito render é impresso para o ecrã a subtração entre tempo final e inicial associado à descrição.

\begin{lstlisting}
void start_time(int num, char* name);
void end_time(int num);
void render();
\end{lstlisting}

