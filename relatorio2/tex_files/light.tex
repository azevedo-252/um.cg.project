Para a iluminação do jogo, foi idealizada uma luz posicionada nas coordenadas do jogador, e com uma intensidade mais fraca que a luz pré-definida. Isto cria um efeito mais sombrio e realista do que a utilização de uma luz uniforme em todo o mapa.

Um dos aspectos mais importantes a ter em conta para a iluminação foi em relação às árvores. Cada árvore é constituida por dois quadrados, mas cada um deles tem uma orientação definida através da sua normal.
Por pré-definição, a iluminação é feita apenas sob as faces frontais de cada poligono, o que faria com que, para cada uma das 4 faces de uma árvore, duas delas não seriam iluminadas.

Para resolver este problema surgiram duas alternativas: utilização de 4 texturas por árvore, cada uma com uma orientação diferente, em vez de apenas duas com a opção GL\_CULL\_FACE desligada para ser desenhada de ambos os lados. Ou em alternativa, a utilização da opção do OpenGL \textbf{glLightModelf(GL\_LIGHT\_MODEL\_TWO\_SIDE, 1)}

Para decidir sobre uma destas implementações, foi utilizado o Profiling, e foram medidos, para cada uma das alternativas, o tempo de render das árvores e o tempo de render total da cena.

De notar que, para serem obtidos números relevantes, foi necessário aumentar o número de árvores para 10000.
Eis os resultados:
\begin{itemize}
\item 2 texturas por árvore, opção GL\_LIGHT\_MODEL\_TWO\_SIDE:
	\begin{itemize}
		\item[render das árvores] 33 ms;
		\item[render total]	49 ms;
		\item[FPS] 20.
	\end{itemize}
\item 4 texturas por árvore, iluminação pré-definida:
	\begin{itemize}
		\item[render das árvores] 85 ms;
		\item[render total]	70;
		\item[FPS] 10. 
	\end{itemize}
\end{itemize}

Como se pode constatar, a primeira opção é claramente a mais eficiente, e portanto foi a solução escolhida.
