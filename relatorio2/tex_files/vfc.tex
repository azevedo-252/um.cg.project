\chapter{View Frsutum Culling}

Da teoria à prática, a maior dificuldade. De entre as diferentes abordagens, após alguma pesquisa, ficamos pela {\bf Geométrica}, pelo simples facto de não haver desvantagens face às restantes a menos em facilidade de implementação.

\section{Bounding Spheres}
Está claro que \textit{axis aligned bounding boxes} é uma melhor implementação que \textit{bounding spheres} uma vez que permite criar uma menor margem de erro para casos de objectos no limite do frustum. No entanto trata-se de um jogo com poucos objectos onde a diferente margem de erro não será significativa e como tal \textit{bounding spheres} são a melhor escolha vez que apresentam o mesmo resultado e mais facil implementação.

Método para teste de \textit{bounding spheres}:
\begin{lstlisting}
int Frustum::sphereInFrustum(Vertex *p, float raio) {

    int result = INSIDE;
    float distance;

    for (int i = 0; i < 6; i++) {
        distance = pl[i].distance(p);
        if (distance < -raio)
            return OUTSIDE;
        else if (distance < raio)
            result = INTERSECT;
    }
    return (result);

}
\end{lstlisting}