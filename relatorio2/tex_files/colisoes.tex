Para aumentar o nível de realismo do jogo, decidimos implementar um sistema de colisões entre o jogador e as torres e árvores, para impedir que este se aproxime demasiado e as atravesse.

As colisões foram implementadadas à custa da seguinte função:

\begin{lstlisting}[caption=Cálculo de colisões]
void Player::calcColisions() {
	//colisoes com as torres
	for (int i = 0; i < g_towers->num_towers; i++) {
		float dist = this->coords->horizontalDistance(g_towers->towers[i]->coords);
		if (dist < tower_colision_dist) {
			float ang = g_towers->towers[i]->ang_x;
			coords->x = (g_towers->towers[i]->coords->x + tower_colision_dist * cos(-ang));
			coords->z = (g_towers->towers[i]->coords->z + tower_colision_dist * sin(-ang));
		}
	}
	
	//colisoes com as arvores
	for (int i = 0; i < g_trees->num_trees; i++) {
		float dist = this->coords->horizontalDistance(g_trees->trees[i]->coords);
		if (dist < tree_colision_dist) {
			Vertex *direction = g_trees->trees[i]->coords->directionVector(this->coords);
			//direction->x = -direction->x;
			//direction->z = -direction->z;
			float ang = acos(direction->x / dist);
			if (this->coords->z <= g_trees->trees[i]->coords->z)
				ang = -ang;
			coords->x = (g_trees->trees[i]->coords->x + tree_colision_dist * cos(-ang));
			coords->z = (g_trees->trees[i]->coords->z + tree_colision_dist * sin(ang));
		}
	}
}
\end{lstlisting}

A colisão apenas faz com que, quando o jogador entra num determinado raio de uma árvore ou torre, é empurrado para fora, na direcção dada pelo angulo entre a torre/árvore e o jogador.
Um aspecto importante a ter é conta é a altura em que esta função é invocada.
Inicialmente era chamada no update do jogador, mas isto fazia com que o jogador bloqueasse sempre que fosse contra uma torre, sendo depois necessário recuar um bocado.
Para resolver este problema, passou-se a fazer a detecção de colisões também depois do update das torres. Desta forma, é permitido que o angulo das torres seja actualizado, durante o seu update, e assim a direcção para a qual o jogador é empurrado já tem em conta o novo angulo, fazendo com que, ao avançar em direcção a uma torre, o jogador automaticamente a contorne.
