As arvores, ao contrário da maioria dos restantes objectos do jogo, não utilizam um modelo 3D. Em vez disso, para simplificar e optimizar o seu render, foi usado um método comum em computação gráfica, que consiste em desenhar dois quadrados, com a textura de uma árvore (e transparências em redor) criando um efeito relativamente aceitável, e com uma performance muito superior, comparativamente a usar um modelo 3D completo.

O número de árvores é configurável no ficheiro ini, e na fase de inicialização, são calculadas posições aleatórias para cada uma delas. Tal como as torres, existe também detecção de colisões entre o jogador e as árvores, não podendo este aproximar-se de uma determinada distância do centro das árvores (distância essa especificada no ficheiro ini).

Para o render das árvores, não haveria nada de relevante a apontar comparativamente aos restantes renders já existentes, não fosse o facto de que neste caso é necessário lidar com transparências. A imagem com a textura usada para a árvore é obviamente rectangular, mas uma parte dessa área não é utilizada, e deve ser ignorada pelo OpenGL, criando um efeito de transparência nessa zona, para que a imagem não apareça com um rectângulo preto a rodeá-la.

Para lidar com isto, utilizou-se o suporte nativo do OpenGL para lidar com o canal Alfa de uma imagem. Isto requer que o formato usado tenha suporte a transparências (foi escolhido o formato TGA), e ainda algum código adicional na fase de render. Eis a função de render das árvores:

\begin{lstlisting}
void Trees::render() {
	glMaterialfv(GL_FRONT_AND_BACK, GL_AMBIENT, mat_amb);
	glMaterialfv(GL_FRONT_AND_BACK, GL_AMBIENT, mat_diff);
	glMaterialfv(GL_FRONT_AND_BACK, GL_SPECULAR, mat_spec);

	g_profiling->start_time(TIME_RENDER_TREES, "Tree Render");
	
	glDisable(GL_CULL_FACE);
	glAlphaFunc(GL_GREATER, 0.15);
	glCallList(treesList);
	glAlphaFunc(GL_ALWAYS, 0);
	glEnable(GL_CULL_FACE);
	
	g_profiling->end_time(TIME_RENDER_TREES);
}
\end{lstlisting}

O render em si está contido na função \textbf{glCallList} pois foi criada uma DisplayList para optimizar o render das árvores. No entanto a parte relevante prende-se com duas instruções:

\begin{itemize}
\item[glDisable(GL_CULL_FACE)] Isto faz com que o OpenGL deixe de ignorar os poligonos quando estes estão virados no sentido oposto ao da câmara. Isto é necessário porque o comportamento desejado é que a textura das árvores seja vista de ambos os lados, para não ser necessário desenhar 4 texturas em vez de duas (mais pormenores na secção de optimização).
\item[glAlphaFunc(GL_GREATER, 0.15)] Esta instrução faz com que o OpenGL ignore todos os próximos dados de desenho cujo valor do campo alfa seja inferior a 0.15. Como a textura TGA tem incluido o valor alfa para cada zona, e as zonas que não correspondem á arvore têm um alfa muito reduzido, isto fará com que estas zonas sejam ignoradas, fazendo com que, de toda a imagem, apenas seja desenhada a parte útil, ficando o resto transparente.
