
O player é aquele que se denomina ``herói'' do jogo. Ainda que nos seja permitido carregar qualquer modelo \textit{md2} optamos pela famosa figura dos jogos electrónicos, \textit{Sonic}.

Para controlar o player criamos uma classe, \textit{Player.cpp}.
Os modelos \textit{md2} utilizados suportam animações por frames. Assim, foi criada a classe Frame para gerir através da função \textbf{glutTimerFunc} o incremento das frames num intervalo fixo de tempo, configurável no ficheiro \textbf{config.ini}. Isto faz com que a cada intervalo, a frame actual seja incrementada, e quando o jogador está em movimento estas alternem ciclicamente, criando a ilusão de movimento.

\-
\begin{tabular} {l | p{10cm}}
\begin{lstlisting}
class Player : public Model_MD2 {
public:
	Frame *anim;
	float ang_x, ang_y;
	float speed_front, speed_back, speed_side;
	float speed_rotate_x, speed_rotate_y;
	float wall_dist;
	bool isJumping;
	int jump_time;
	int jump_max;
	bool canJump;
	int jump_cooldown;
	int tower_colision_dist;
	int tree_colision_dist;

	GameState state;

	Player(const std::string &path);

	void move(Vertex *new_coords);
	bool isMoving();
	void update();
	void render();

	float jumpOff(int off);

	static void inc_frame(int val);

	void calcColisions();
};
\end{lstlisting} 
&
Header da classe Player.cpp .\\
\end{tabular}
