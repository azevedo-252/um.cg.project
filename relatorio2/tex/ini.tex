

Um ponto relevante do projecto passa pelo ficheiro {\bf config.ini} . Neste documento estão todas as variáveis de uso frequente, ou melhor, todas as variáveis manipuláveis sem implicar recompilar o jogo, o que seria necessário para valores em \#define. Estes valores são "carregados"\ para o jogo quando necessário, mariotariamente no inicar do mesmo.

\-
\begin{center}
\begin{tabular} {l | p{10cm}}
\begin{lstlisting}
[game]
distance_factor		= 4
updates_per_second	= 60
anims_per_second	= 20
num_towers			= 15
num_keys			= 1
tower_range			= 1000
towers_min_distance = 500
\end{lstlisting} 
& 
Este é um exemplo de algumas configurações do jogo no documento {\bf config.ini}. \\
\end{tabular}
\end{center}

Este processo simplifica dois aspectos muito importantes. Para começar, possibilita a existência de um \emph{repositório} central com todos os valores usados ao longo do jogo. Por outro, facilita a alteração destes valores, o que ajuda bastante a efectuar pequenos ajustes no jogo sem ter a necessidade de recompilar o programa (processo que, como se constatou, tornou-se bastante longo á medida que o tamanho do código foi crescendo).

Eis de seguida um exemplo da utilidade desta classe, neste caso na criação da instância do jogador:

\begin{lstlisting}
Player::Player(const string &path) : Model_MD2(path) {
	state = GAME_ON;
	coords = new Vertex();
	coords->x = GLManager::distance(conf.rfloat("player:x"));
	coords->z = GLManager::distance(conf.rfloat("player:z"));
	coords->y = g_map->triangulateHeight(coords->x, coords->z);
	direction = new Vertex(0, 0, 1);
	ang_x = 0;
	ang_y = M_PI / 2;
	md2_rendermode = 0;

	speed_front = GLManager::convertFromKmH(conf.rfloat("player:speed_front"));
	speed_back = GLManager::convertFromKmH(conf.rfloat("player:speed_back"));
	speed_side = GLManager::convertFromKmH(conf.rfloat("player:speed_side"));
	speed_rotate_x = conf.rfloat("player:speed_rotate_x");
	speed_rotate_y = conf.rfloat("player:speed_rotate_y");


	anim = new Frame();
	anim->add_anim(MOVE_NONE, conf.rint("player:stop_frame"), conf.rint("player:stop_frame_end"));
	anim->add_anim(MOVE_WALK, conf.rint("player:walk_frame"), conf.rint("player:walk_frame_end"));
	anim->add_anim(MOVE_JUMP, conf.rint("player:jump_frame"), conf.rint("player:jump_frame_end"));

	jump_max = conf.rint("player:jump_max");
	isJumping = false;
	jump_time = 0;
	canJump = true;
	jump_cooldown = conf.rint("player:jump_cooldown");

	tower_colision_dist = conf.rint("player:tower_colision_dist");
	tree_colision_dist = conf.rint("player:tree_colision_dist");
}
\end{lstlisting}

Como se pode ver, esta classe depende fortemente de valores registados no ini, que são aqui lidos através das funções \textbf{conf.rint} e \textbf{conf.rfloat}, demonstrando assim a utilidade deste processo.
